\documentclass[12pt]{article}
\usepackage[margin=1in]{geometry}
\usepackage{amsmath, amssymb}
\usepackage{enumitem}

\begin{document}

\begin{center}
{\Large \textbf{In-Class Exercise (Canvas Quiz) --- 10 minutes}}\\[6pt]
{\large \textbf{Lecture 12: GMMs and Performance Evaluation}}
\end{center}

\vspace{8pt}

We will evaluate clustering outputs using performance metrics.

\vspace{6pt}
\noindent
\textbf{Given:} The following clustering results from two methods on five samples, along with ground truth (GT):

\vspace{6pt}
\begin{center}
\begin{tabular}{c c c c c}
\hline
Sample & Features & GT Cluster & Method 1 & Method 2 \\
\hline
1 & [0,1] & 1 & 1 & 1 \\
2 & [1,1] & 1 & 1 & 2 \\
3 & [3,1] & 2 & 2 & 2 \\
4 & [5,1] & 2 & 3 & 3 \\
5 & [6,1] & 3 & 3 & 3 \\
\hline
\end{tabular}
\end{center}

\vspace{8pt}
\noindent
\textbf{Round numerical answers to two decimals.}

\begin{enumerate}[label=(\arabic*), itemsep=10pt]

\item Which Rand Index values are correct?

\begin{itemize}
\item[(A)] $RI_1=0.60,\; RI_2=0.80$
\item[(B)] $RI_1=0.80,\; RI_2=0.60$
\item[(C)] $RI_1=1.00,\; RI_2=0.80$
\end{itemize}

\item Which Purity values are correct?

\begin{itemize}
\item[(A)] $P_1=0.60,\; P_2=0.80$
\item[(B)] $P_1=0.80,\; P_2=0.60$
\item[(C)] $P_1=1.00,\; P_2=0.80$
\end{itemize}

\item If ground-truth labels are \textbf{unknown}, which metric is most appropriate?

\begin{itemize}
\item[(A)] Rand Index
\item[(B)] Silhouette Coefficient
\item[(C)] Purity
\end{itemize}

\item Which statement describes a limitation of \textbf{Purity}?

\begin{itemize}
\item[(A)] It penalizes having too many clusters.
\item[(B)] It can increase artificially when the number of clusters increases.
\item[(C)] It does not require ground truth labels.
\end{itemize}

\item What does a \textbf{high silhouette score} indicate?

\begin{itemize}
\item[(A)] Clusters are overlapping and poorly separated.
\item[(B)] Points are close to their own cluster and far from other clusters.
\item[(C)] The number of clusters equals the number of data points.
\end{itemize}

\end{enumerate}

\end{document}
