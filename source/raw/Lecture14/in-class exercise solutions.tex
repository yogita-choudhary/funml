\documentclass[12pt]{article}
\usepackage[margin=1in]{geometry}
\usepackage{amsmath, amssymb}
\usepackage{enumitem}

\begin{document}

\begin{center}
{\Large \textbf{In-Class Exercise --- Solutions}}\\[6pt]
{\large \textbf{Lecture 14: Convolution and Pooling Output Sizes}}
\end{center}

\vspace{6pt}
We use:
\[
H_{\text{out}}=\left\lfloor \frac{H_{\text{in}}-k_h+2P}{S} \right\rfloor + 1,
\qquad
W_{\text{out}}=\left\lfloor \frac{W_{\text{in}}-k_w+2P}{S} \right\rfloor + 1.
\]
Assume all operations are valid (i.e., the result is an integer).

\vspace{6pt}
\noindent\textbf{Given:}
\[
(H_{\text{in}},W_{\text{in}},C_{\text{in}})=(64,64,3),\quad
(k_h,k_w,C_{\text{in}})=(8,8,3),\ S=4,\ P=0.
\]
Pooling:
\[
(k_h,k_w)=(4,4),\ S=1,\ P=0.
\]

\vspace{8pt}
\begin{enumerate}[label=(\arabic*), itemsep=12pt]

\item \textbf{Multiple Choice (select all that apply).}
In a convolution layer, which of the following \emph{directly} affects the output \emph{spatial} dimensions $(H_{\text{out}},W_{\text{out}})$?
\begin{itemize}
\item[(A)] Number of filters (output channels)
\item[(B)] Stride $S$
\item[(C)] Padding $P$
\item[(D)] Kernel size $(k_h,k_w)$
\end{itemize}

\textbf{Solution.}
Spatial size depends on the formula, so it is controlled by $S$, $P$, and $(k_h,k_w)$. The number of filters changes the output \emph{channels}, not $(H_{\text{out}},W_{\text{out}})$.
\[
\boxed{(B),(C),(D)}
\]

\item \textbf{Multiple Choice.}
After the convolution layer, what is the output spatial size $(H_{\text{conv}},W_{\text{conv}})$?
\begin{itemize}
\item[(A)] $(14,14)$
\item[(B)] $(15,15)$
\item[(C)] $(16,16)$
\item[(D)] $(56,56)$
\end{itemize}

\textbf{Solution.}
\[
H_{\text{conv}}=\frac{64-8+2(0)}{4}+1=\frac{56}{4}+1=15,
\qquad
W_{\text{conv}}=15.
\]
\[
\boxed{(B)}
\]

\item \textbf{Multiple Choice.}
After the max-pooling layer, what is the output spatial size $(H_{\text{pool}},W_{\text{pool}})$?
\begin{itemize}
\item[(A)] $(11,11)$
\item[(B)] $(12,12)$
\item[(C)] $(15,15)$
\item[(D)] $(19,19)$
\end{itemize}

\textbf{Solution.}
Pooling takes $(15,15)$ as input:
\[
H_{\text{pool}}=\frac{15-4+2(0)}{1}+1=12,
\qquad
W_{\text{pool}}=12.
\]
\[
\boxed{(B)}
\]

\item \textbf{True/False.}
Keeping $H_{\text{in}},W_{\text{in}},k_h,k_w,S$ fixed, increasing padding $P$ in convolution increases the output spatial size.

\textbf{Solution.}
Padding increases the effective input size ($H_{\text{in}}+2P$, $W_{\text{in}}+2P$), which increases the output spatial dimensions when other quantities are fixed.
\[
\boxed{\text{True}}
\]

\end{enumerate}

\end{document}
