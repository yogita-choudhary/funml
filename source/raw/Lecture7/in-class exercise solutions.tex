\documentclass[12pt]{article}
\usepackage[margin=1in]{geometry}
\usepackage{amsmath, amssymb}
\usepackage{enumitem}

\begin{document}

\begin{center} 
{\Large \textbf{In-Class Exercise --- Solutions}}\\[6pt]
{\large \textbf{Lecture 7: Linear Regression (Least Squares / Normal Equation)}}
\end{center}

\vspace{8pt}
\section*{Question 1}

We use the linear regression model:
\[
\boldsymbol{\hat{y}} = \mathbf{X}\boldsymbol{\hat{\theta}},
\qquad
\boldsymbol{\hat{\theta}} = (\mathbf{X}^T \mathbf{X})^{-1}\mathbf{X}^T \boldsymbol{y}
\]

\textbf{Given:}
\[
\mathbf{X} =
\begin{bmatrix}
1 & 1 & 0\\
0 & 2 & 1\\
1 & 0 & 1
\end{bmatrix},
\qquad
\boldsymbol{y} =
\begin{bmatrix}
2\\
3\\
4
\end{bmatrix}
\]

\subsection*{Solutions}

\begin{enumerate}[label=(\arabic*), itemsep=10pt]

\item
Compute the transpose matrix $\mathbf{X}^T$.
\[
\mathbf{X}^T =
\begin{bmatrix}
1 & 0 & 1\\
1 & 2 & 0\\
0 & 1 & 1
\end{bmatrix}
\]

\item
Compute the matrix $\mathbf{X}^T \mathbf{X}$.

\[
\mathbf{X}^T \mathbf{X} =
\begin{bmatrix}
2 & 1 & 1\\
1 & 5 & 2\\
1 & 2 & 2
\end{bmatrix}
\]

\item

Compute the vector $\mathbf{X}^T \boldsymbol{y}$.

\[
\mathbf{X}^T \boldsymbol{y} =
\begin{bmatrix}
6\\
8\\
7
\end{bmatrix}
\]

\item

State the normal equation expression for the least squares solution $\boldsymbol{\hat{\theta}}$.

\[
\boldsymbol{\hat{\theta}} = (\mathbf{X}^T \mathbf{X})^{-1}\mathbf{X}^T \boldsymbol{y}
\]

\item

Compute the final least squares solution $\boldsymbol{\hat{\theta}}$.

\[
\boldsymbol{\hat{\theta}} =
\begin{bmatrix}
5/3\\
1/3\\
7/3
\end{bmatrix}
\approx
\begin{bmatrix}
1.67\\
0.33\\
2.33
\end{bmatrix}
\]

% \item

\item Compute the predicted output for
\[
\boldsymbol{x}_{\text{new}} =
\begin{bmatrix}
1\\
1\\
1
\end{bmatrix},
\qquad
\hat{y}_{\text{new}} = \boldsymbol{x}_{\text{new}}^T \boldsymbol{\hat{\theta}}.
\]

\[
\hat{y}_{\text{new}} =
\begin{bmatrix}1 & 1 & 1\end{bmatrix}
\boldsymbol{\hat{\theta}}
= 4.33
\]

\end{enumerate}

\vspace{12pt}

\section*{Question 2 — Correct Matches}

Below are 5 scenarios involving linear regression. \textbf{Column A} shows two quantities for each scenario. \textbf{Column B} shows the corresponding other two quantities, but \textbf{in shuffled order}.

\noindent Match each row in Column A to the correct row in Column B by using logical reasoning, dimension checking, and simple mental arithmetic. \textbf{Do NOT compute the normal equations}. Instead, verify relationships like $\boldsymbol{\hat{y}} = \mathbf{X}\boldsymbol{\hat{\theta}}$.

\begin{center}
\renewcommand{\arraystretch}{2.5}
\begin{tabular}{|c|p{6cm}||c|p{6cm}|}
\hline
\multicolumn{2}{|c||}{\textbf{Column A}} & \multicolumn{2}{c|}{\textbf{Column B}} \\
\hline\hline
\textbf{A1} & $\mathbf{X} = \begin{bmatrix} 1 & 0 \\ 0 & 1 \end{bmatrix}, \quad \boldsymbol{y} = \begin{bmatrix} 2 \\ 3 \end{bmatrix}$ 
& \textbf{B1} & $\mathbf{X} = \begin{bmatrix} 1 \\ 1 \\ 1 \end{bmatrix}, \quad \boldsymbol{y} = \begin{bmatrix} 4 \\ 5 \\ 6 \end{bmatrix}$ \\
\hline
\textbf{A2} & $\mathbf{X} = \begin{bmatrix} 2 \\ 3 \end{bmatrix}, \quad \boldsymbol{\hat{\theta}} = \begin{bmatrix} 2 \end{bmatrix}$ 
& \textbf{B2} & $\boldsymbol{\hat{\theta}} = \begin{bmatrix} 2 \\ 3 \end{bmatrix}, \quad \boldsymbol{\hat{y}} = \begin{bmatrix} 2 \\ 3 \end{bmatrix}$ \\
\hline
\textbf{A3} & $\boldsymbol{\hat{\theta}} = \begin{bmatrix} 5 \end{bmatrix}, \quad \boldsymbol{\hat{y}} = \begin{bmatrix} 5 \\ 5 \\ 5 \end{bmatrix}$ 
& \textbf{B3} & $\mathbf{X} = \begin{bmatrix} 1 \\ 2 \\ 3 \end{bmatrix}, \quad \boldsymbol{\hat{\theta}} = \begin{bmatrix} 2 \end{bmatrix}$ \\
\hline
\textbf{A4} & $\boldsymbol{y} = \begin{bmatrix} 2 \\ 4 \\ 6 \end{bmatrix}, \quad \boldsymbol{\hat{y}} = \begin{bmatrix} 2 \\ 4 \\ 6 \end{bmatrix}$ 
& \textbf{B4} & $\boldsymbol{\hat{\theta}} = \begin{bmatrix} 1 \\ 2 \end{bmatrix}, \quad \boldsymbol{\hat{y}} = \begin{bmatrix} 3 \\ 5 \\ 7 \end{bmatrix}$ \\
\hline
\textbf{A5} & $\mathbf{X} = \begin{bmatrix} 1 & 1 \\ 1 & 2 \\ 1 & 3 \end{bmatrix}, \quad \boldsymbol{y} = \begin{bmatrix} 3 \\ 5 \\ 7 \end{bmatrix}$ 
& \textbf{B5} & $\boldsymbol{y} = \begin{bmatrix} 4 \\ 6 \end{bmatrix}, \quad \boldsymbol{\hat{y}} = \begin{bmatrix} 4 \\ 6 \end{bmatrix}$ \\
\hline
\end{tabular}
\end{center}

\[
\boxed{
A1 \to B2,\quad
A2 \to B5,\quad
A3 \to B1,\quad
A4 \to B3,\quad
A5 \to B4
}
\]

\begin{center}
\Large
\begin{tabular}{cl}
\textbf{A1} & $\to$ \textbf{B2} \\[6pt]
\textbf{A2} & $\to$ \textbf{B5} \\[6pt]
\textbf{A3} & $\to$ \textbf{B1} \\[6pt]
\textbf{A4} & $\to$ \textbf{B3} \\[6pt]
\textbf{A5} & $\to$ \textbf{B4} \\
\end{tabular}
\end{center}

\vspace{12pt}
\hrule
\vspace{12pt}

\subsection*{Detailed Explanations}

\noindent
\textbf{A1 $\to$ B2:}

Given: $\mathbf{X} = \begin{bmatrix} 1 & 0 \\ 0 & 1 \end{bmatrix}$ (identity matrix), $\mathbf{y} = \begin{bmatrix} 2 \\ 3 \end{bmatrix}$

Match: $\boldsymbol{\hat{\theta}} = \begin{bmatrix} 2 \\ 3 \end{bmatrix}, \boldsymbol{\hat{y}} = \begin{bmatrix} 2 \\ 3 \end{bmatrix}$

\noindent With identity matrix $\mathbf{X} = \mathbf{I}$, we have $\boldsymbol{\hat{y}} = \mathbf{I}\boldsymbol{\hat{\theta}} = \boldsymbol{\hat{\theta}}$. For perfect fit, $\boldsymbol{\hat{\theta}} = \mathbf{y}$. Dimensions: $\mathbf{X}$ is $2 \times 2$, so $\boldsymbol{\hat{\theta}}$ must be $2 \times 1$.

\vspace{12pt}

\noindent
\textbf{A2 $\to$ B5:}

Given: $\mathbf{X} = \begin{bmatrix} 2 \\ 3 \end{bmatrix}$, $\boldsymbol{\hat{\theta}} = \begin{bmatrix} 2 \end{bmatrix}$

Match: $\mathbf{y} = \begin{bmatrix} 4 \\ 6 \end{bmatrix}, \boldsymbol{\hat{y}} = \begin{bmatrix} 4 \\ 6 \end{bmatrix}$

\noindent Compute $\boldsymbol{\hat{y}} = \mathbf{X}\boldsymbol{\hat{\theta}} = \begin{bmatrix} 2 \\ 3 \end{bmatrix} \cdot 2 = \begin{bmatrix} 4 \\ 6 \end{bmatrix}$. Perfect fit means $\mathbf{y} = \boldsymbol{\hat{y}}$.

\vspace{12pt}

\noindent
\textbf{A3 $\to$ B1:}

Given: $\boldsymbol{\hat{\theta}} = \begin{bmatrix} 5 \end{bmatrix}$, $\boldsymbol{\hat{y}} = \begin{bmatrix} 5 \\ 5 \\ 5 \end{bmatrix}$

Match: $\mathbf{X} = \begin{bmatrix} 1 \\ 1 \\ 1 \end{bmatrix}, \mathbf{y} = \begin{bmatrix} 4 \\ 5 \\ 6 \end{bmatrix}$

\noindent  Constant predictions ($\boldsymbol{\hat{y}}$ all 5's) require $\mathbf{X}$ to be a column of ones. Verify: $\begin{bmatrix} 1 \\ 1 \\ 1 \end{bmatrix} \cdot 5 = \begin{bmatrix} 5 \\ 5 \\ 5 \end{bmatrix}$. Dimensions match: $3 \times 1$ matrix.

\vspace{12pt}

\noindent
\textbf{A4 $\to$ B3:}

Given: $\mathbf{y} = \begin{bmatrix} 2 \\ 4 \\ 6 \end{bmatrix}$, $\boldsymbol{\hat{y}} = \begin{bmatrix} 2 \\ 4 \\ 6 \end{bmatrix}$

Match: $\mathbf{X} = \begin{bmatrix} 1 \\ 2 \\ 3 \end{bmatrix}, \boldsymbol{\hat{\theta}} = \begin{bmatrix} 2 \end{bmatrix}$

\noindent Need $\mathbf{X}\boldsymbol{\hat{\theta}} = \boldsymbol{\hat{y}}$. Check: $\begin{bmatrix} 1 \\ 2 \\ 3 \end{bmatrix} \cdot 2 = \begin{bmatrix} 2 \\ 4 \\ 6 \end{bmatrix}$ ✓. The pattern [1,2,3] scaled by 2 gives [2,4,6].

\vspace{12pt}

\noindent
\textbf{A5 $\to$ B4:}

Given: $\mathbf{X} = \begin{bmatrix} 1 & 1 \\ 1 & 2 \\ 1 & 3 \end{bmatrix}$, $\mathbf{y} = \begin{bmatrix} 3 \\ 5 \\ 7 \end{bmatrix}$

Match: $\boldsymbol{\hat{\theta}} = \begin{bmatrix} 1 \\ 2 \end{bmatrix}, \boldsymbol{\hat{y}} = \begin{bmatrix} 3 \\ 5 \\ 7 \end{bmatrix}$

\noindent Dimensions: $\mathbf{X}$ is $3 \times 2$, so $\boldsymbol{\hat{\theta}}$ must be $2 \times 1$. Verify:
\[
\begin{bmatrix} 1 & 1 \\ 1 & 2 \\ 1 & 3 \end{bmatrix} \begin{bmatrix} 1 \\ 2 \end{bmatrix} 
= \begin{bmatrix} 1 + 2 \\ 1 + 4 \\ 1 + 6 \end{bmatrix}
= \begin{bmatrix} 3 \\ 5 \\ 7 \end{bmatrix} \quad 
\]
Perfect fit with $\mathbf{y} = \boldsymbol{\hat{y}}$.


\vspace{12pt}

\noindent One can verify via calculations as follows: 

A1-B2: $\boldsymbol{\hat{y}} = \mathbf{X} \boldsymbol{\hat{\theta}} = \begin{bmatrix} 1 & 0 \\ 0 & 1 \end{bmatrix}\begin{bmatrix} 2 \\ 3 \end{bmatrix} = \begin{bmatrix} 2 \\ 3 \end{bmatrix}$

A2-B5: $\boldsymbol{\hat{y}} = \mathbf{X} \boldsymbol{\hat{\theta}} = \begin{bmatrix} 2 \\ 3 \end{bmatrix}\begin{bmatrix} 2  \end{bmatrix} = \begin{bmatrix} 4 \\ 6 \end{bmatrix}$

A3-B1: $\boldsymbol{\hat{y}} = \mathbf{X} \boldsymbol{\hat{\theta}} = \begin{bmatrix} 1 \\ 1 \\ 1 \end{bmatrix} \begin{bmatrix} 5 \end{bmatrix} = \begin{bmatrix} 5 \\ 5 \\ 5 \end{bmatrix}$

A4-B3: $\boldsymbol{\hat{y}} = \mathbf{X} \boldsymbol{\hat{\theta}} = \begin{bmatrix} 1 \\ 2 \\ 3 \end{bmatrix} \begin{bmatrix} 2 \end{bmatrix} = \begin{bmatrix} 2 \\ 4 \\ 6 \end{bmatrix}$

A5-B4: $\boldsymbol{\hat{y}} = \mathbf{X} \boldsymbol{\hat{\theta}} = \begin{bmatrix} 1 & 1 \\ 1 & 2 \\ 1 & 3 \end{bmatrix} \begin{bmatrix} 1 \\ 2 \end{bmatrix} = \begin{bmatrix} 3 \\ 5 \\ 7 \end{bmatrix}$

\end{document}
