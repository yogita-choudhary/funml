\documentclass[12pt]{article}
\usepackage[margin=1in]{geometry}
\usepackage{amsmath, amssymb}
\usepackage{enumitem}

\begin{document}

\begin{center}
{\Large \textbf{In-Class Exercise (Canvas Quiz) --- 15 minutes}}\\[6pt]
{\large \textbf{Lecture 7: Linear Regression (Least Squares / Normal Equation)}}
\end{center}

\vspace{8pt}
\section*{Question 1}

We use the linear regression model:
\[
\boldsymbol{\hat{y}} = \mathbf{X}\boldsymbol{\hat{\theta}},
\qquad
\boldsymbol{\hat{\theta}} = (\mathbf{X}^T \mathbf{X})^{-1}\mathbf{X}^T \boldsymbol{y}
\]

\textbf{Given:} Consider the following dataset with 3 data points and 3 features:
\[
\mathbf{X} =
\begin{bmatrix}
1 & 1 & 0\\
0 & 2 & 1\\
1 & 0 & 1
\end{bmatrix},
\qquad
\boldsymbol{y} =
\begin{bmatrix}
2\\
3\\
4
\end{bmatrix}
\]

\textbf{Tasks:}
\begin{enumerate}[label=(\arabic*), itemsep=8pt]

\item Compute the transpose matrix $\mathbf{X}^T$.

\item Compute the matrix $\mathbf{X}^T \mathbf{X}$.

\item Compute the vector $\mathbf{X}^T \boldsymbol{y}$.

\item State the normal equation expression for the least squares solution $\boldsymbol{\hat{\theta}}$.

\item Compute the final least squares solution $\boldsymbol{\hat{\theta}}$.

\item Compute the predicted output for
\[
\boldsymbol{x}_{\text{new}} =
\begin{bmatrix}
1\\
1\\
1
\end{bmatrix},
\qquad
\hat{y}_{\text{new}} = \boldsymbol{x}_{\text{new}}^T \boldsymbol{\hat{\theta}}.
\]

\end{enumerate}

\textbf{Round all final numeric answers to 2 decimals. A calculator is allowed (and recommended).}

\vspace{14pt}

%%%%%%%%%%%%%%%%%
%%%%%%%%%%%%%%%%%
%%%%%%%%%%%%%%%%%

\newpage 
\section*{Question 2}

Below are 5 scenarios involving linear regression. \textbf{Column A} shows two quantities for each scenario. \textbf{Column B} shows the corresponding other two quantities, but \textbf{in shuffled order}.

\noindent Match each row in Column A to the correct row in Column B by using logical reasoning, dimension checking, and simple mental arithmetic. \textbf{Do NOT compute the normal equations}. Instead, verify relationships like $\boldsymbol{\hat{y}} = \mathbf{X}\boldsymbol{\hat{\theta}}$.

\begin{center}
\renewcommand{\arraystretch}{2.5}
\begin{tabular}{|c|p{6cm}||c|p{6cm}|}
\hline
\multicolumn{2}{|c||}{\textbf{Column A}} & \multicolumn{2}{c|}{\textbf{Column B}} \\
\hline\hline
\textbf{A1} & $\mathbf{X} = \begin{bmatrix} 1 & 0 \\ 0 & 1 \end{bmatrix}, \quad \boldsymbol{y} = \begin{bmatrix} 2 \\ 3 \end{bmatrix}$ 
& \textbf{B1} & $\mathbf{X} = \begin{bmatrix} 1 \\ 1 \\ 1 \end{bmatrix}, \quad \boldsymbol{y} = \begin{bmatrix} 4 \\ 5 \\ 6 \end{bmatrix}$ \\
\hline
\textbf{A2} & $\mathbf{X} = \begin{bmatrix} 2 \\ 3 \end{bmatrix}, \quad \boldsymbol{\hat{\theta}} = \begin{bmatrix} 2 \end{bmatrix}$ 
& \textbf{B2} & $\boldsymbol{\hat{\theta}} = \begin{bmatrix} 2 \\ 3 \end{bmatrix}, \quad \boldsymbol{\hat{y}} = \begin{bmatrix} 2 \\ 3 \end{bmatrix}$ \\
\hline
\textbf{A3} & $\boldsymbol{\hat{\theta}} = \begin{bmatrix} 5 \end{bmatrix}, \quad \boldsymbol{\hat{y}} = \begin{bmatrix} 5 \\ 5 \\ 5 \end{bmatrix}$ 
& \textbf{B3} & $\mathbf{X} = \begin{bmatrix} 1 \\ 2 \\ 3 \end{bmatrix}, \quad \boldsymbol{\hat{\theta}} = \begin{bmatrix} 2 \end{bmatrix}$ \\
\hline
\textbf{A4} & $\boldsymbol{y} = \begin{bmatrix} 2 \\ 4 \\ 6 \end{bmatrix}, \quad \boldsymbol{\hat{y}} = \begin{bmatrix} 2 \\ 4 \\ 6 \end{bmatrix}$ 
& \textbf{B4} & $\boldsymbol{\hat{\theta}} = \begin{bmatrix} 1 \\ 2 \end{bmatrix}, \quad \boldsymbol{\hat{y}} = \begin{bmatrix} 3 \\ 5 \\ 7 \end{bmatrix}$ \\
\hline
\textbf{A5} & $\mathbf{X} = \begin{bmatrix} 1 & 1 \\ 1 & 2 \\ 1 & 3 \end{bmatrix}, \quad \boldsymbol{y} = \begin{bmatrix} 3 \\ 5 \\ 7 \end{bmatrix}$ 
& \textbf{B5} & $\boldsymbol{y} = \begin{bmatrix} 4 \\ 6 \end{bmatrix}, \quad \boldsymbol{\hat{y}} = \begin{bmatrix} 4 \\ 6 \end{bmatrix}$ \\
\hline
\end{tabular}
\end{center}

\end{document}
