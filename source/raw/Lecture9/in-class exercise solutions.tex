\documentclass[12pt]{article}
\usepackage[margin=1in]{geometry}
\usepackage{amsmath, amssymb, bm}
\usepackage{enumitem}
\begin{document}

\begin{center}
{\Large \textbf{In-Class Exercise — Solutions}}\\[6pt]
{\large \textbf{Lecture 9: Regularization and Performance Metrics}}
\end{center}

\vspace{6pt}
\textbf{Given}
\[
\bm{y} =
\begin{bmatrix} 1 \\ 2 \\ 2 \end{bmatrix},
\qquad
\hat{\bm{y}} =
\begin{bmatrix} 1 \\ 1 \\ 3 \end{bmatrix},
\qquad N=3
\]

\[
\bm{\theta}=
\begin{bmatrix} 2 \\ -1 \end{bmatrix},
\qquad
\|\bm{\theta}\|_2^2 = 5,
\qquad
\gamma=0.60
\]

\[
L(\bm{\theta})
= \frac{1}{N}\sum_{i=1}^N(\hat{y}_i-y_i)^2
+ \frac{\gamma}{N}\|\bm{\theta}\|_2^2
\]

\begin{enumerate}[label=(\arabic*), itemsep=18pt]

\item \textbf{Question:}  
What is the residual vector $\bm{r}=\hat{\bm{y}}-\bm{y}$?

\begin{itemize}
\item[(A)] $\begin{bmatrix}0\\1\\1\end{bmatrix}$
\item[(B)] $\begin{bmatrix}0\\-1\\1\end{bmatrix}$
\item[(C)] $\begin{bmatrix}1\\-1\\1\end{bmatrix}$
\end{itemize}

\textbf{Solution:}
\[
\bm{r}=\hat{\bm{y}}-\bm{y}
=\begin{bmatrix}1-1\\1-2\\3-2\end{bmatrix}
=\begin{bmatrix}0\\-1\\1\end{bmatrix}
\]
\[
\boxed{(B)}
\]

\item \textbf{Question:}  
What is the Mean Squared Error (MSE)?

\begin{itemize}
\item[(A)] $0.33$
\item[(B)] $0.67$
\item[(C)] $1.00$
\end{itemize}

\textbf{Solution:}
\[
\mathrm{MSE}=\frac{1}{3}(0^2+(-1)^2+1^2)=\frac{2}{3}\approx0.67
\]
\[
\boxed{(B)}
\]

\item \textbf{Question:}  
What is the ridge objective value $L(\bm{\theta})$?

\begin{itemize}
\item[(A)] $0.67$
\item[(B)] $1.00$
\item[(C)] $1.67$
\end{itemize}

\textbf{Solution:}
\[
\frac{\gamma}{N}\|\bm{\theta}\|_2^2=\frac{0.60}{3}\cdot5=1.00
\]
\[
L(\bm{\theta})=0.6667+1.00=1.6667\approx1.67
\]
\[
\boxed{(C)}
\]

\item \textbf{Question:}  
Suppose the third prediction changes from $3$ to $8$.  
Which metric would increase the most?

\begin{itemize}
\item[(A)] MAE
\item[(B)] RMSE
\item[(C)] MSE
\end{itemize}

\textbf{Solution:}
Squared error amplifies large mistakes the most.
\[
\boxed{(C)}
\]

\item \textbf{Question:}  
Two models have the same MSE, but Model A has much larger $\|\bm{\theta}\|_2^2$.  
Which model does ridge regression prefer?

\begin{itemize}
\item[(A)] Model A (larger weights)
\item[(B)] Model B (smaller weights)
\item[(C)] Both equally
\end{itemize}

\textbf{Solution:}
Ridge penalizes large parameter magnitudes.
\[
\boxed{(B)}
\]

\item \textbf{Question:}  
If $\gamma$ increases, what happens to the learned weights?

\begin{itemize}
\item[(A)] Weights grow larger
\item[(B)] Weights shrink toward zero
\item[(C)] Weights stay unchanged
\end{itemize}

\textbf{Solution:}
\[
\boxed{(B)}
\]

\end{enumerate}

\end{document}
