\documentclass[12pt]{article}
\usepackage[margin=1in]{geometry}
\usepackage{amsmath, amssymb, bm}
\usepackage{enumitem}

\begin{document}

\begin{center}
{\Large \textbf{In-Class Exercise --- Solutions}}\\[6pt]
{\large \textbf{Lecture 11: Clustering (k-Means)}}
\end{center}

\vspace{8pt}

\noindent
\textbf{Given centroids:}
\[
\bm{m}_1 = (2,3),\quad \bm{m}_2 = (7,8),\quad \bm{m}_3=(5,2)
\]
\textbf{New data point:}
\[
\bm{x}_{\text{new}} = (4,4)
\]
Tie-breaking rule: if there is a tie, choose the cluster with the \textbf{smallest index}.

\begin{enumerate}[label=(\arabic*), itemsep=18pt]

\item \textbf{Question:} Which statement is correct?
\begin{itemize}
\item[(A)] Minimizing $d(\bm{x},\bm{m}_j)$ is different from minimizing $d(\bm{x},\bm{m}_j)^2$.
\item[(B)] Minimizing $d(\bm{x},\bm{m}_j)$ is equivalent to minimizing $d(\bm{x},\bm{m}_j)^2$.
\item[(C)] Squaring the distance changes the nearest centroid whenever distances are $<1$.
\end{itemize}

\textbf{Solution:}
Since squaring is a strictly increasing function on nonnegative numbers, the $\arg\min$ is unchanged.
\[
\boxed{(B)}
\]

\item \textbf{Question:} Compute squared distances. Which option matches $(d_1^2,d_2^2,d_3^2)$?
\begin{itemize}
\item[(A)] $(5,\;25,\;5)$
\item[(B)] $(25,\;5,\;5)$
\item[(C)] $(5,\;20,\;4)$
\end{itemize}

\textbf{Solution:}
\[
d_1^2=(4-2)^2+(4-3)^2=4+1=5
\]
\[
d_2^2=(4-7)^2+(4-8)^2=9+16=25
\]
\[
d_3^2=(4-5)^2+(4-2)^2=1+4=5
\]
\[
\boxed{(A)}
\]

\item \textbf{Question:} Which cluster should $\bm{x}_{\text{new}}$ be assigned to?
\begin{itemize}
\item[(A)] Cluster 1
\item[(B)] Cluster 2
\item[(C)] Cluster 3
\end{itemize}

\textbf{Solution:}
We have a tie: $d_1^2=d_3^2=5$. Tie-breaking picks the smallest index.
\[
\boxed{(A)}
\]

\item \textbf{Question:} If Cluster 1 contains $\bm{x}^{(1)}=(2,3)$ and $\bm{x}^{(2)}=(4,4)$, what is the updated centroid $\bm{m}_1$?
\begin{itemize}
\item[(A)] $(3,3.5)$
\item[(B)] $(3,4)$
\item[(C)] $(4,3.5)$
\end{itemize}

\textbf{Solution:}
Centroid is the mean of coordinates:
\[
\bm{m}_1=\left(\frac{2+4}{2},\frac{3+4}{2}\right)=(3,3.5)
\]
\[
\boxed{(A)}
\]

\item \textbf{Question:} Which condition indicates k-Means has converged?
\begin{itemize}
\item[(A)] Centroids stop changing (or change is negligible), so assignments also stop changing.
\item[(B)] Centroids keep changing but assignments stop changing.
\item[(C)] Assignments keep changing but centroids stop changing.
\end{itemize}

\textbf{Solution:}
Convergence is reached when the algorithm stabilizes (no meaningful change in centroids / no reassignment).
\[
\boxed{(A)}
\]

\item \textbf{Question:} Which statement is true about k-Means?
\begin{itemize}
\item[(A)] k-Means always finds the global optimum regardless of initialization.
\item[(B)] k-Means can converge to a poor local solution depending on initialization.
\item[(C)] k-Means is not affected by outliers because it uses Euclidean distance.
\end{itemize}

\textbf{Solution:}
k-Means depends on initialization and can get stuck in local minima; it is also sensitive to outliers.
\[
\boxed{(B)}
\]

\end{enumerate}

\end{document}
