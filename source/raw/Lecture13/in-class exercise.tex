\documentclass[12pt]{article}
\usepackage[margin=1in]{geometry}
\usepackage{amsmath, amssymb, bm}
\usepackage{enumitem}

\begin{document}

\begin{center}
{\Large \textbf{In-Class Exercise (Canvas Quiz) --- 10 minutes}}\\[4pt]
{\large \textbf{Lecture 13: Neural Networks}}
\end{center}

\vspace{6pt}
\noindent Answer all questions. Show work on scratch paper if helpful.

\begin{enumerate}[label=(\arabic*), itemsep=10pt]

\item \textbf{Forward pass with ReLU}\\
A neuron computes $h=\bm{w}^T\bm{x}+b$ and outputs $\mathrm{ReLU}(h)=\max(0,h)$.\\
Given
\[
\bm{x}=\begin{bmatrix}1\\-2\end{bmatrix},\quad
\bm{w}=\begin{bmatrix}2\\1\end{bmatrix},\quad
b=-1,
\]
what is the output after ReLU?

\begin{itemize}
\item[(A)] $0$
\item[(B)] $1$
\item[(C)] $3$
\end{itemize}

\item \textbf{Why add nonlinearities?}\\
What is the main reason we include nonlinear activation functions (e.g., ReLU, sigmoid) in neural networks?

\begin{itemize}
\item[(A)] They make the optimization problem convex.
\item[(B)] They allow the network to represent non-linear functions/decision boundaries.
\item[(C)] They guarantee perfect generalization.
\end{itemize}

\item \textbf{Softmax sanity check}\\
Softmax maps scores $\bm{s}=[s_1,s_2,s_3]$ to
\[
p_k=\frac{e^{s_k}}{\sum_{j=1}^3 e^{s_j}}.
\]
Which statement is always true?

\begin{itemize}
\item[(A)] Some $p_k$ can be negative.
\item[(B)] $\sum_{k=1}^3 p_k = 1$.
\item[(C)] The largest score always maps to probability exactly $1$.
\end{itemize}

\item \textbf{Cross-entropy intuition}\\
If the true class is class 2, the loss is $L=-\log(p_2)$. Which prediction gives the smallest loss?

\begin{itemize}
\item[(A)] $p_2 = 0.90$
\item[(B)] $p_2 = 0.40$
\item[(C)] $p_2 = 0.10$
\end{itemize}

\item \textbf{Counting parameters}\\
Fully-connected network:
\[
\text{Input }(4)\rightarrow \text{Hidden }(3)\rightarrow \text{Output }(2),
\]
with biases in each layer. How many trainable parameters (weights + biases) are there?

\begin{itemize}
\item[(A)] $4\cdot3 + 3\cdot2$
\item[(B)] $4\cdot3 + 3 + 3\cdot2 + 2$
\item[(C)] $4 + 3 + 2$
\end{itemize}

\end{enumerate}

\end{document}
